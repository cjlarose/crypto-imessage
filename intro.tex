Every day millions of text messages are sent around the world.  The senders of
these messages would like to think that the messages they send will be read
only by their intended recipient but just how secure are these messages really?
The speed and power of the modern computer has birthed a new necessity for
powerful encryption algorithms that are impervious to technological attacks---
algorithms which make it so that if your message is intercepted it would still
take an attacker decades, or even centuries, to decrypt it by force.  Apple,
the manufacturer of one of the most popular cellular devices on the planet, has
devised their own cryptographic system to protect the messages of their
consumers.  Many of their products run iOS and offer the iMessage service which
allows iOS users to freely send text and picture messages to other users who
use iMessage services.  Apple claims that these iMessages are secure and that
not even they could decrypt an intercepted iMessage.  From the information that
Apple has made public, we can see that the entire process of sending an iMessage
utilizes a complex amalgamation of cutting-edge cryptographic techniques.  It
is the purpose of this paper to explore the cryptographic algorithm used in
iMessage, especially Transport Layer Security and the Elliptic Curve Digital
Signature Algorithm, as well the hardware that facilitates their use in order
to better understand how Apple achieves a high level security of their users.
