Secure Sockets Layer (SSL) and Transport Layer Security (TLS) are security
protocols for establishing a secure connection between a client and a server,
typically a web server.  This is useful to hide a clients private information,
such as credit card information or a social security number, from the public.
SSL is the predecessor of TLS and was developed by Netscape Communications
Corporation in 1994.  Later the Internet Engineering Task Force (IETF)
standardized SSL 3.0 to form TLS 1.0, in 1999.  A secure connection between the
client and server is accomplished by both asymmetric and symmetric key
cryptography through a process of key exchanges, client/server authentication,
and algorithm/cipher agreements.

The process of establishing a secure connection between client and server is
done through a series of messages between the two.  It begins with the client,
and a "Client Hello" message.  This message is sent to the server with a list
of client supported ciphers and a randomly generated number that will be used
later.  After the server receives the Client Hello message, it begins
comprising the "Server Hello" message.  This message includes a selected cipher
to be used on the next message, another randomly generated number, and a
certification.  The certification is a way of authenticating the server to the
client or vise versa and is normally issued by a certification authority (CA).
A CA is a mutually trusted third party that verifies the identity of the
requestor; once confirmed, the CA comprises the requestors information as long
with a public key in a message to be sent (Oppliger 105).  Once the Server
Hello message is received by the client, it confirms the CA that issued the
certification is on its list of trusted CAs.  Once verified, the client then
extracts the servers public key from the certification and comprises the
"Client Key Exchange" message by creating a random symmetric key and encrypts
the key using the server's public key (normally using RSA) and sends the
message to the server.  The server then decrypts the symmetric key by using its
private key and now both parties have the symmetric key.  Before security is
concluded, one last step is taken.  Each side sends a MAC or HMAC (message
authentication code or hash message authentication code) using the random
numbers sent and the symmetric key.  MAC and HMAC are hash functions that
cannot be reversed engineered, since the functions are non-injective.  As well
as sending the MAC or HMAC, the client sends another cipher to be used on the
future messages.  This final step ensures that the information throughout the
process is constant and that there has not been an attack on the information
passed.  Once each sides checks and verifies the MAC or HMAC received, secure
communication is concluded and the two can begin sending information securely
(Rescorla 58).   

Since SSL/TLS was developed, many attacks have succeeded in breaking its
security.  Meyer and Schwenk point out in their paper, that the protocol for
SSL/TLS should be followed and documented rigorously during implementation, and
that most of the weakness are a result of poor implementation(14).  A majority
of the weakness with SSL/TLS follow from a man in the middle attack on the
Client Hello and Server Hello messages.  Because the protocol allows for the
server to pick out of a list of available client supported ciphers, an attacker
could edit this list and remove the ciphers with higher security.  This would
force the server to pick a weaker cipher or to not allow communication.  Recent
attacks have involved this process and the RC4 stream cipher in particular.
Overall, with proper implementation of the protocol, and a strong list of
ciphers,  one can assume secure communication with SSL/TLS.

Acknowledgements:
Meyer, Christopher, and Jorg Schwenk. "Lessons Learned From Previous SSL/TLS Attacks A Brief Chronology Of Attacks And Weaknesses." eprint.iacr.org. Ruhr-University, n.d. Web. 22 Apr. 2014. <https://eprint.iacr.org/2013/049.pdf>.
Oppliger, Rolf. SSL and TLS theory and practice. Boston: Artech House, 2009. Print.
Rescorla, Eric. SSL and TLS: designing and building secure systems. Boston: Addison-Wesley, 2001. Print.
