\subsection{Elliptic Curve Theory}

Elliptic curves are complicated geometric and algebraic objects,
whose full explanation lies outside the scope of this paper.
The purpose of this section is to prove a vague understanding
of why elliptic curve cryptography might be secure, so that the reader
can have a top level understanding of how the iMessage protocol
works, and is secure.\\

An elliptic curve is an equation in the form $y^2 = x^3 + ax^2 + bx + c$.

When working with elliptic curves, it is often most useful to describe the curve as a set
$$\{(x,y) \vert x,y \in K, y^2 = x^3 + ax^2 + bx + c\} \cup \{ \infty \}$$, where $K$ is a set such as
$\mathbb{Q}, \mathbb{R}, \mathbb{Z} \mod p$, where $p$ is prime.

We include a point at
$\infty$, which serves as the identity under point addition.
Conceptually, we think of as lines through $\infty$
as vertical lines.

The elliptic curves with cryptographic applications are ones where $K$ is the integers
mod a prime $p$, denoted $\mathbb{F}_p$.

\begin{example}
Consider the elliptic curve $E = y^2 = x^3+2x-1 \mod 5$.

Using the set notation, we can describe this curve as
$E = \{(0 ,2 ),(0,3),(2,1),(2 ,4 ),(4,1),(4,4), \infty\}$.
\end{example}




\subsubsection{Adding Points of an Elliptic Curve}
Suppose we want to add points $P_1$ and $P_2$.

The following algorithm serves as addition on elliptic curve points.
\begin{center}
\begin{verbatim}
Given an elliptic curve E with points P1 and P2
if P1 == P2:
    let L be the line tangent to E intersecting P1
else:
    let L be the line through both P1 and P2
Let R =/= P1,P2 be the third point that L intersects on E
Suppose R = (rx, ry)
Then P1 + P2 = (rx, -ry)
\end{verbatim}
\end{center}

The existence of such a point $R$ is given by the properties of elliptic curves,
so this operation is well defined.

As mentioned previously, there is an $\infty$ point that serves as the additive identity.

Consider $(x,y) + \infty$.
The line through $(x,y)$ and $\infty$ will be completely vertical.
So the point of intersection (labeled $R$ in the algorithm) will be $(x,-y)$.
Then, following the algorithm, we get that $(x,y) + \infty = (x,y)$.




\subsection{The Discrete Logarithm Problem}

\subsubsection{The classic Discrete Log Problem}
A variant of the discrete logarithm problem is the foundation of elliptic
curve cryptography.  The classic discrete logarithm problem is a follows:\\

Consider an arbitrary group $(G, *)$ and suppose $b,g \in G$.

The discrete logarithm problem is to find an integer $k$
such that $b^k = g$.

Finding such a $k$ is computationally difficult; yet
With $k$ known, it is very simple to compute $b^k$.

Since exponentiation is the easy part of an invertible function, with
the logarithm being difficult to compute, this problem has cryptographic
applications.


\subsubsection{The Discrete Logarithm Problem for Elliptic Curves}

Consider an elliptic curve $E$ with points $A,B$.

The elliptic curve discrete Logarithm problem to find $k$
such that $B = kA$.

Analogous to the classic discrete logarithm problem, finding such a $k$
is computationally difficult, but computing $B$ is relatively simple.

The discrete logarithm problem for elliptic curves is the foundation of
elliptic curve cryptography.



\subsection{DSA with Elliptic Curves}












