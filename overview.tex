tep of encryption begins at the sender's iOS device when the device generates two key pairs.  The first is a 1280-bit key used for encrypting the message itself using an RSA encryption algorithm.  The second key is a 256-bit ECDSA key used for signing the message so the receiver can verify the message was sent from the correct device.  Both RSA encryption and ECDSA encryption belong to a class of encryption algorithms known as public-key encryption. In a public-key cryptosystem two pairs of keys are generated.  One of the keys is known as the public key.  This key is used for encrypting the plaintext, or the actual message itself, but it does so in such a way that it is effectively impossible to use the public key to decrypt the message.  This property of the public key means it poses no security risk for others to know of the key, hence the name public.  In contrast, the private key can only be used to decrypt ciphertext that has been encrypted with the corresponding public key.  Thus, the private key must be only known to the intended receiver of the message in order to guarantee security of the message.  Returning to iOS, when the two key pairs are generated the device stores the private keys locally, in a secure area of the device.  It then sends the two public keys to the Apple Push Notification Service server, APNs, where they are associated with the device's phone number.  The public keys of any receiver can then be pushed down to any device wanting to send an iMessage to another iOS device.


The second step is actually sending the iMessage which begins by adding recipients of the iMessage.  The sender's iOS device finds the phone numbers of the recipients in its local Contact list and contacts the APNs server, requesting the public keys associated with the phone numbers of the recipients.  This contact with the server is the first vulnerability we see so far.  Both the server and the sender's device need to know that their communication is uninterrupted.  It is possible for the sender's device to send a message to a third-party, malicious receiver who, upon reading the message, then forwards the information along to the APNs server so it looks like there was no interupption of communication.  Such an attack is known as a Man-in-the-middle attack due to the malicious third party in the middle of the communication.  To ensure a secure communication between the device and APNs, communication is encrypted using an algorithm known as Transport Layer Security, or TLS.
