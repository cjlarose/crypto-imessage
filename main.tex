\documentclass[11pt,titlepage,]{article}
\usepackage{amsmath,amssymb, amsthm}
\usepackage{array}
\usepackage{graphicx}
\usepackage{algorithm} %ctan.org\pkg\algorithms
\usepackage{algpseudocode}
\usepackage{mathtools}
\usepackage{float}
\usepackage{cite}
\usepackage{hyperref}
\usepackage[abbrev,msc-links]{amsrefs}
\usepackage{fullpage}

\parindent 0em
\parskip 1em

\newcommand\inputfile[1]{
        \InputIfFileExists{#1}{}{\typeout{No file #1.}}%
}


\newtheoremstyle{indenteddefinition}{.5\topsep}{.5\topsep}{\addtolength{\leftskip}{2.5em}}{-2.5em}{\bfseries}{}{\newline}{}
\theoremstyle{indenteddefinition}
\newtheorem{example}{Example}[section]

% This is how you define your own macros.
\newcommand{\ZZ}{\mathbb{Z}}

\title{Applications in Cryptography: Security of iMessage}
\author{Daniel Belcher, Matt Gautreau, Chris LaRose, Taylor Siebenberg}
\date{Introduction to Cryptography\\University of Arizona\\Spring 2014}

\begin{document}

\maketitle

\section*{Introduction}\label{sec:intro} % asterisk causes section to be unnumbered
\inputfile{intro.tex}

\section{Hardware Support for Cryptography}\label{sec:hardward}
\inputfile{hardware.tex}

\section{Elliptic Curve Digital Signature Algorithm (ECDSA)}\label{sec:ecdsa}
\inputfile{ecdsa.tex}

\section{Secure Sockets Layer (SSL) \& Transport Layer Security (TLS)}\label{sec:tls}
\inputfile{SSL_TLS.tex}

\section{Apple iMessage Encryption Protocol}\label{sec:overview}
\inputfile{overview.tex}

\newpage
\inputfile{bib.tex}

\end{document}

