The cryptographic strength of iMessage relies in part on a number of security
features present in every iOS device. Cryptographic algorithms are often
complex and can present significant implications in performance and battery
life.  These priorities drive the architecture of iOS device security to help
make implementations of cryptographic algorithms fast and safe.

\subsection{AES Coprocessor}
% AES 256 engine, SHA-1 in hardware
At the core of Apple's security architecture is a hardware device referred to
as the Crypto Engine or the AES Coprocessor. During the manufacturing process,
two 256-bit AES Keys are burned into the coprocessor. The first is the unique
ID (UID), which, as its name implies, is unique to every iOS device. Apple
states that the UID is not recorded by Apple or any of its
suppliers\cite{apple}. The second key is a device group ID (GID), which is
shared by all iOS devices of the same class (all devices with an Apply A7
processor share the same GID, for example).

The GID key is used to decrypt firmware images. This means that Apple can
update low-level pieces of the device architecture (like the bootloader or
kernel) securely, providing safety for end users when installing these updates.

The UID is used to generate a number of other AES keys (called \textit{derived
keys}) by enciphering some hardcoded values with the UID at boot time. For
example, \texttt{0x01010101010101010101010101010101} is enciphered using the
UID as a key to produce a File Data Protection key. This derived key used to
decrypt file metadata, which contains another file-specific key, which is
ultimately used to decrypt a file. Further details on this scheme, however, lie
outside of the scope of this paper.

Burning the UID and GID keys into the silicon itself helps the keys from being
accessed from outside of the Crypto Engine. Essentially, the coprocessor is
used to provide the output of cryptographic operations without exposing the
keys themselves.

\subsection{Random Number Generation}
% CTR_DRBG
% System entropy from hardware
Pseudorandom number generation plays an important role in many cryptographic
operations, such as salts generated for use in ECDSA (discussed later). iOS
devices use a random number generator (RNG) that uses an algorithm based on
Counter mode Deterministic Random Byte Generator (CTR\_DRBG).  CTR\_DRBG is one
of four cryptographically secure pseudorandom number generators published by
the National Institute of Standards and Technology in NIST SP
800-90A\cite{nistsp80090a}.

CSPRNGs generally fall into three classes: those based on cryptographic
primitives (ciphers and hashes), those of number theoretic designs (like Blum
Blum Shub), and those of special-purpose designs. Of the four CSPRNGs of NIST
SP 800-90A, three are of the type based on cryptographic primitives: Hash\_DRBG
(based on hash functions), HMAC\_DRBG (based on hash-based message
authentication code), and CTR\_DRBG (based on block ciphers). The fourth CSPRNG
in 800-90A is Dual\_EC\_DRBG, is of the number theoretic class and is based on
elliptic curve cryptography. It was later reported to likely contain a backdoor
inserted by the National Security Agency\cite{goodin}. The other RNGs,
including CTR\_DRBG, however, are still considered to be secure.

CTR\_DRBG is based on running a block cipher in counter
mode\cite{nistsp80090a}. Apple does not specify which block cipher is used in
their implementation, but NIST specifies that CTR\_DRBG be used only with an
\textit{approved} cipher. Currently, NIST approves only AES, Triple DES, and
Skipjack for these purposes\cite{nistblockciphers}, so it is presumed that
Apple uses one of these. CTR\_DRBG is initialized with a random key and input
data and encrypts the data with the provided cipher and key in counter mode,
returing the enciphered bytes. In iOS devices, the CTR\_DRBG is seeded and
reseeded with information gathered from ``interrupt timing during boot, and
additionally from internal sensors once the device has booted''\cite{apple}.

\subsection{Effaceable Storage}
% Sucurity implications of flash storage
Generating keys securely is just one half of the equation: it is also important
to make sure that keys are ultimately destroyed from storage. iOS devices use
flash storage, a type of non-volatile storage. Flash storage provides a number
of benefits iOS devices because drives start up almost instantaneously, can
handle higher temperatures than traditional hard disks, and there are no moving
parts. There are, however, some drawbacks for using flash storage for
cryptographic purposes. Bytes can be individually read or writen to in with
random access, but they can only be erased (set back to 1s) one ``block'' a
time. This means that to erased cryptographic keys from storage, one has to
erase the entire block in which it resides. Fundamentally, this is exactly how
iOS erases keys.

Effaceable Storage is a single small portion of the device's NAND flash storage
used specifically for the storage of cryptographic keys. When the device is
powered off, keys in this segment of storage are wiped. The same process occurs
if users request a remote wipe of their device. This means that even if
an attacker has physical access to the device after it's been turned off or
wiped, the data on the device would be undecipherable\cite{garfinkel}. This
mechanism provides great security for end users in case their device falls into
the hands of an attacker. It also has the effect, as Garfinkel notes, of
empowering criminals. The data stored on iOS devices is so secure that even
forensic investigators trying to access data on a suspect's device often have
no way to access it outside of brute-forcing keys, which, for AES-256, is
practically impossible.
